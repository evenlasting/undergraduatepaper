\cleardoublepage
\chapter{中期检查}

\section{项目概况}
故事线是一种表达数据集中不同实体关系的可视化方式,比如电影\cite{Rmovie}、社交数据集\cite{liu2013storyflow}、源代码\cite{gansner1993technique}等。从定义上说,故事线可视化使用线条来表达角色和情节之间的关系。故事线研究者通过一些优化模型\cite{liu2013storyflow},比如贪心算法、优化方程等,来生成数据集的故事线布局。近年来故事线可视化技术已经取得较好的进展,可以自动生成复杂故事的插图,但是其在效率,交互集合设计上还存在问题,同时在开源社区中没有任何一个故事线生成库来供大家调用。
针对这些问题,本项目的目标有三个:
\begin{itemize}
    \item 通过缩减优化方程,加速StoryFlow\cite{liu2013storyflow}的故事线生成算法。并将加速后的算法作为npm包发布。
    \item 调研故事线生成中的必要交互,在iStoryline\cite{tang2019istoryline}的基础上对于交互空间做出优化。使其可以支持从头开始的故事线创作。
    \item 开发一个系统,应用以上两个功能,并部署在服务器上。
\end{itemize}

本项目最主要价值就是对故事线可视化的开源社区做出一些贡献。

我的工作是处理故事线生成算法的优化,npm包的开发,整个系统的调试与部署。本系统的前端,将由同实验室的其他同学配合完成。

\section{工作进展情况}
算法优化部分的设计已经完成,原算法为:
\begin{equation}
    \label{equ:midCheck}
\min \sum_{i=1}^{n_{e}} \sum_{j}^{n_{t}-1}\left(y_{i, j}-y_{i, j+1}\right)^{2}+\beta \sum_{i=1}^{n_{e}} \sum_{j=1}^{n_{t}} y_{i, j}^{2}
\end{equation}

新的算法将引入相对位置固定的故事线块的概念,使用块作为优化的基本单位。从而缩小优化方程的计算量。npm包还在开发中,整体架构设计已经完成。其将整个工作流程分为:1.排序,2.对齐,3.压缩,4.自由变换。每个流程将提供不同的方法,以供用户自由组合,生成自己更喜欢的生产线。代码开发还未完成,正在稳步进行中。

调研故事线的必要交互在开题时已经完成。我们发现艺术家在创作故事线时一定会使用到一个交互:添加角色。但是这在当前的故事线系统交互集合中是缺失的。因为当前的所有故事线系统都是从故事脚本出发,而不能修改故事脚本。这和他们的系统分前后端有一定的关系。他们不倾向于改变发送到后端的故事脚本。所以之前的系统是不能自由创作的。这是不自然的一种设计。

因为我们将会使用自己设计的npm包进行系统开发,所以可以对于没有初始脚本的故事进行创作。这才能让手绘故事线的作者沉浸其中,更还原手绘的感觉。但又给予作者一些布局提醒、设计帮助。此交互集合的设计也已经完成。我们在iStoryline\cite{tang2019istoryline}的交互集合中引入了对于角色和情节做增删改的交互。

同时,系统前端已经由同组的其它同学开发完毕,我通过docker和k8将其部署在了之江实验室的gitlab服务器上。但是其使用的后端并不是此设计最终提交的版本,使用npm包替代后端的工作将在下个月进行。

\section{问题与建议}
从开题到中期遇到的最大困难在部署服务器上:
\begin{itemize}
    \item 对于ci/cd技术不熟悉,不会使用docker以及k8工具
    \item 对于部署中遇到的跨域问题不熟悉
\end{itemize}

这些问题都可以通过仔细阅读文档解决。

其次,因为疫情影响,不能使用实验室熟悉的电脑的开发环境,配环境十分难受。但是这个问题在学会了docker之后迎刃而解,我可以随时拉下来别人配好的环境进行开发工作。

接下来是建议方面。对于个人发展,我会希望自己在提交最终文档时,是使用latex进行编写的,而不是偷懒地使用word。
对于专业方向,因为顶会的论文都必须有明显的进步,所以他们不方便对于现有的问题进行小修小补。而本设计将会完善故事线设计上的多个环节:开源、算法优化以及交互优化。这也是毕设的优势吧,可以做小而有益的事情。
对于实验室,实验室的学长学姐们最近在赶每年一次的顶会,没有什么建议,希望他们顺利就好。
对于学院建设,我希望能对于学生提供一些服务器,至少学校要搭建一个gitlab来方便学生部署项目以及课程作业。这样我也就不需要之江实验室的服务器来进行部署了。

\section{其他}
其它部分就说说在这一个月中开发的想法吧。首先是开源的问题,我觉得论文作为之后人员研究的参考,一定要开源,并且配套发布装有相应环境的docker。就算论文的思想写得再详细,算法中的细枝末节也不是后辈一下子就能体会到的。不开源的论文就等于不想让别人参考,非常不利于科研的发展。

然后是对于现在开发环境发展的赞美,比如docker,比如前端中的各种手脚架。这些东西对于开发体验的提高太大了。希望学校能在各种课程中多介绍最新的开发环境,让同学们用最近最好的东西,而不是拘泥于十年前,二十年前的过时技术。
