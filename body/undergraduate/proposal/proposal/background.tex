\section{问题提出的背景}
故事线可视化技术已经取得较好的进展,可以自动生成复杂故事的插图,但是尽管在性能和拓展方面有所改进,故事情节的视觉布局并没有得到改变,始终较难逼近手绘故事情节。
\subsection{背景介绍}
故事是指对时间和因果序列的叙述。每个故事都有开始,发展和结束。故事还涉及一个或多个实体,实体决定了故事的情节,也就是实体之间的交互。分析实体之间发生的交互在很多情况下是重要的,比如社交网络分析\cite{liu2013storyflow}、讲故事\cite{Rmovie}以及媒体分析\cite{tanahashi2015efficient}。例如,给定一系列与“新冠”有关系的推文,一位社会科学家可能会对这些话题中的意见领袖(实体)之间的交锋(交互)和结果产生兴趣。而故事线可视化就可以帮助用户更好地理解和分析复杂的故事。
在故事线可视化之中为了正确的表达角色之间的关系,我们应该遵循这样的设计准则\cite{ogawa2010software}:
\begin{itemize}
    \item 定义1 同一个情节的线条应该上下相邻
    \item 定义2 不同情节的线条应该尽量远离
    \item 定义3 一个线条在不变化情节时应该尽量水平
\end{itemize}
同时为了产生更美观和紧凑的布局,我们应该遵循这样的美学原则\cite{ogawa2010software}:
\begin{itemize}
    \item 原则1 减少线的弯折
    \item 原则2 减少线的交叉
    \item 原则3 减少空白区域
\end{itemize}
符合这样的原则的故事线会有足够的美观性和可读性。但是太过扁平,缺少了特定的叙事隐喻,也就是缺少了叙事性。艺术家创造的手绘故事线被认为是具有叙事性的故事线样本。在自动生成的故事线和手绘故事线之间一直有着叙事性的沟壑\cite{tang2019istoryline}。

在自动生成故事线的优化算法发展到瓶颈之后,近年来故事线研究的重心逐步从提出定义、提出优化方法,转移到如何通过人机交互的方式创造出富有表现力的故事线。
\subsection{本研究的意义和目的}
传统的故事线生成算法\cite{liu2013storyflow}分为三步:
\begin{itemize}
    \item 使用贪心进行排序
    \item 使用动态规划进行对齐
    \item 使用解优化方程获得坐标
\end{itemize}

第三步需要解一个庞大的优化方程,解这个方程的速度限制了整个生成的速度。对于有规模的故事,无类型语言(JavaScript)实现的生成算法无法实时产生结果。这限制了故事线生成算法不能作为一个前端包发布。

本研究将对这个庞大的优化方程进行缩减,使得故事线创作可以作为一个独立的前端包发布,可以简单应用在所有的可视化系统之中。
同时本研究也将进行调研了解艺术家是如何创作手绘故事线的,对于iStoryline\cite{tang2019istoryline}提出的设计空间进行优化。完善其交互,使得用户可以更好的实现自己的想法,表达出自己的叙事想法,提高故事线的美学品质。

本研究的意义如下:
\begin{itemize}
    \item 通过缩小优化方程的规模加速故事线生成算法,使得可以以Javascript发布完整的故事线生成包
    \item 对手绘故事线设计空间进行调研,通过调研得到的设计空间提出高效灵活的交互框架,使用上述包搭建对应的框架设计
\end{itemize}





