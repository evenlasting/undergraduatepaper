\section{项目的主要内容和技术路线}
论文的主要内容将分为两个部分,其一缩小优化方程,其二调研手绘故事线设计空间,设计交互框架。这两个部分虽然不相关,但都是为了最终搭建前端框架而服务。
\subsection{主要研究内容}
主要研究内容也将分为两个方面来叙述,其一从哪个方向来优化约束方程,其二如何优化目前的iStoryline交互框架。
\subsubsection{缩小优化方程}
根据美观原则,我们的故事布局要更紧凑,也就是空白要最小化。

布局中的空白只存在于角色之间,以及角色和画布边框之间。要使得这些空白最小,我们可以表达成一个二次优化问题\cite{liu2013storyflow}。其中e代表实体,t代表时间,y是坐标,β是平衡这两种空白的参数。
\begin{equation}
    \min \sum_{i=1}^{n_{e}} \sum_{j}^{n_{t}-1}\left(y_{i, j}-y_{i, j+1}\right)^{2}+\beta \sum_{i=1}^{n_{e}} \sum_{j=1}^{n_{t}} y_{i, j}^{2}
\end{equation}

但是显然,这个问题的解不能是简单的全0,所以这个二次优化问题需要有一些约束,这些约束就构建在前面两步的排序和对齐之中,下面将简单介绍解优化之前的两步。

\textbf{排序}

为了在满足美观原则,我们要让线的交叉最小。所以我们在每个时间范围内对于代表实体的线进行垂直排序。但是根据故事线的定义,同一个情节里面线必须相近。这样这个问题就变得复杂起来了。为了解决这个问题,我们对于排序问题分为两个部分情节之内实体的顺序,情节之间的顺序。

对于情节之间的顺序,我们使用贪心算法,实体较多的情节优先安放,实体较少的情节根据交叉数最少原则,插入目前最好的位置。

接下来,这个问题就变成了适用DAG扫描算法\cite{liu2013storyflow}的问题,我们从第一帧扫描到最后一帧,再来回扫描直到稳定,或者到达最大迭代数。最终我们得到了一个$O(n_en_tlog_{n_e})$的算法,其可以求出交叉较少的一种排序。

\textbf{对齐}

为了尽可能减少弯曲的次数,我们尽可能对齐较多的线段,也就是给他们分配相同的y坐标。这个约束可以被表示为:
\begin{equation}
    E_{\text {align}}=\max \sum_{t=1}^{n_{t}-1} H(t)
\end{equation}

其中H(t)是t到t+1之间的直线段数。

对齐主要包括两个部分:情节间的对齐和情节内角色的对齐。而这两种对齐问题都可以转化为LCS(最长公共子序列)\cite{liu2013storyflow}问题,我们只需要定义一个相似函数来进行状态转移即可。在StoryFlow中,这个相似函数被定义为:
\begin{equation}
    \operatorname{sim}\left(l_{i}, r_{j}\right)=\operatorname{straight}\left(l_{i}, r_{j}\right)+\alpha\left(1-\left|\frac{i}{m}-\frac{j}{n}\right|\right)
\end{equation}

第一部分是li到rj之间最大的不弯折角色数,第二部分衡量两个情节的在时间点中的相对位置。

使用这种方法,我们就可以在$O({n_e}^2n_t)$的时间内得到较优的一组对齐解.

在这两步之后,我们就获得了以下四个优化方程的约束:
\begin{equation}
\begin{array}{ll}
    y_{i_{1}, j}<y_{i_{2}, j}, & \text { if } S_{i_{1}, j} \prec S_{i_{2}, j} \\
    y_{i, j}=y_{i, j+1}, & \text { if } S_{i, j} \leftrightarrow S_{i, j+1} \\
    y_{i, j}-y_{i+1, j}=d_{\text {in }}, & \text { if } \operatorname{SID}\left(S_{i, j}\right)=\operatorname{SID}\left(S_{i+1, j}\right) \\
    \left|y_{i, j}-y_{i+1, j}\right| \geq d_{\text {out }}, & \text { if } \operatorname{SID}\left(S_{i, j}\right) \neq \operatorname{SID}\left(S_{i+1, j}\right)
    \end{array}    
\end{equation}

如何缩减这个优化方程呢?我们可以有这样的基本思路,目前是将实体的每一个时间作为一个变量,再通过拉直来进行约束。但是实际上,我们可以将每个拉直的部分作为一个变量来进行求解。这样将大幅度缩小变量空间,以达到加速的目的。
\subsubsection{调研交互框架}
我们已经对于艺术家创作故事线的整个历程进行了调研,其中至少会包含一个操作:添加角色。

但是在过去的交互框架中从来不会包含这样的操作,这是为什么呢?为了添加角色会更改脚本内容,出于保持简单的考虑,之前的研究都未对于这样的交互框架进行研究。而本研究将基于从头创作故事线进行研究,并完善之前的基于改善机器生成故事线的交互框架。

同时之前的框架还有一个巨大的问题,他们只能针对情节进行交互,而不能对于更细粒度的时间来进行操作。这个问题的原因也是出于,此前的研究只是为了帮助研究者把机器生成的故事线逼近手绘故事线,我们这个研究才将真正调研使用什么样的交互框架是最有利于从头创作故事线的。

这个调研步骤将从艺术家的创作出发,我们将总结艺术家在从头创作故事线时的必要操作,如何能将这些操作抽象为高阶交互,从而减少艺术家的故事线创作时间。真正的创作时从头开始的,艺术家的灵感也和科研一样,是在创作时迸发的,而不是一开始就有一个完整的想法,然后再生成机器故事线,而后通过交互去逼近自己的想法。
\subsection{技术路线}
对于缩小优化方程规模的研究部分,这是一个建模问题。如果我们将变量设置为拉直的部分,可以简单推导出这样的带线性约束的二次优化方程:
\begin{equation}
    \label{equ:pop_com}
    \min \sum_{i=1}^{n_{p}}\left(p_{i}-p_{i, \text {adjoin}}\right)^{2}+\beta \sum_{i=1}^{n_{p}}\left(p_{i}\right)^{2}
  \end{equation}

其中p代表一个拉直的部分,pi,adjoin代表与此部分上下相邻的拉直部分。这个优化方程的约束和原式的约束相近,这里不再赘述.

对于调研的研究部分,这是一个用户研究问题。需要注意的是怎样的问卷设计可以全面获得艺术家的真实创作需求。以及怎么的交互才可以全面而简洁的覆盖这个需求。

整个研究的技术部分主要在于最终框架的编写以及故事线创作包的发布。最终框架会基于react,使用发布的npm故事线创作包,来进行编写。

其中画布的渲染会使用paper.js库来进行,这个库可以很好的兼容svg和canvas。最终的技术路线会如\autoref{fig:pop_struct}所示
\begin{figure}[ht]
    \centering
    \includegraphics[width=0.3\linewidth]{pop/struct}
    \caption{\label{fig:pop_struct}框架结构}
  \end{figure}

\subsection{可行性分析}
本文内容已在暑期实践中进行了大部分,在github中的iStoryline.js仓库即可访问,整体框架已经搭建完毕,目前有10个star,有可视化领域研究者进行了简单试用并评价。这说明本研究是可行的,可完成的。

作为试用,我们目前可以使用此仓库创作简单的故事线,在完全基于前端的前提下,此仓库可以达到实时交互的目的。也即初步达到了我们的需求。不过还需要进一步完善。

进一步完善的内容包括目前提出的系统问题修复,这些修复都是基于代码层面的,而不是算法层面,是可以在一定时间内逐步排查的。

进一步完善的内容还包括交互设计,本研究对于艺术家创作中的痛点收集已经完成,基本交互已经全部完成,在之后将进行迭代开发,逐步完善交互框架设计。

