\cleardoublepage

\section{绪论}
故事线可视化在呈现复杂实体关系方面广受欢迎。它可以创作出吸引人的视觉元素。全自动故事线可视化技术目前已经取得较好的进展,可以快速生成复杂故事的插图。但是尽管在性能方面进步很快,全自动方法生成的视觉布局无法得到改变,难以逼近手绘故事情节。而能逼近手绘布局的半自动方法一直只存在于论文中,没有向大众开放的软件。
\subsection{故事线研究背景}
故事是指对时间和因果序列的叙述。每个故事都有开始,发展和结束。故事还涉及一个或多个实体,实体决定了故事的情节,也就是实体之间的交互。分析实体之间发生的交互在很多情况下是重要的,比如社交网络分析\cite{liu2013storyflow}、电影介绍\cite{Rmovie}以及媒体分析\cite{tanahashi2015efficient}。例如,给定一系列与“新冠”有关系的推文,一位社会科学家可能会对这些话题中的用户(实体)之间的对话(交互)和结果产生兴趣。而故事线可视化就可以帮助用户更好地理解和分析复杂的故事。

在故事线可视化之中为了正确的表达角色之间的关系,我们应该遵循这样的设计准则\cite{ogawa2010software}:
\begin{itemize}
    \item 定义1 同一个情节的线条应该上下相邻
    \item 定义2 不同情节的线条应该尽量远离
    \item 定义3 一个线条在不变化情节时应该尽量水平
\end{itemize}
同时为了产生更美观和紧凑的布局,我们应该遵循这样的美学原则\cite{ogawa2010software}:
\begin{itemize}
    \item 原则1 减少线的弯折
    \item 原则2 减少线的交叉
    \item 原则3 减少空白区域
\end{itemize}
符合这样的原则的故事线会有足够的美观性和可读性。但是太过扁平,缺少了特定的叙事隐喻,也就是缺少了叙事性。艺术家创造的手绘故事线被认为是具有叙事性的故事线样本。在自动生成的故事线和手绘故事线之间一直有着叙事性的沟壑\cite{tang2019istoryline}。

在自动生成故事线的优化算法发展到瓶颈之后,近年来故事线研究的重心逐步从提出定义、提出优化方法,转移到如何通过人机交互的半自动方式创造出富有表现力的故事线。但是半自动化方式始终只停留在论文中,没有面相大众的编辑工具。

\subsection{故事线研究意义}
传统的故事线生成算法\cite{liu2013storyflow}分为三步:
\begin{itemize}
    \item 使用贪心进行排序
    \item 使用动态规划进行对齐
    \item 使用解优化方程获得坐标
\end{itemize}
第三步需要解一个庞大的优化方程,解这个方程的速度限制了整个故事线生成的速度。对于规模较大的故事,无法快速解出方程就会影响用户的交互体验。

本设计将使用一种新的槽结构对这个庞大的优化方程进行缩减,可以让用户获得更好的体验。本设计也使用拓扑排序和并查集对排序做了数据结构层面的优化,小幅度提高第一步的处理速度。

同时本研究也将对于iStoryline\cite{tang2019istoryline}提出的设计空间进行优化。增加渲染和变形模块,完善其交互空间,使得用户可以更好的实现自己的想法,表达出自己的叙事思路,提高故事线的美学品质。

我也对于两位图形学研究者进行了用户调研,证明了此系统的易用性以及生成故事线的美观程度,叙事程度。我们最后还给出了一个使用本系统进行复现正义联盟故事的创作例子,来证明此系统的可用性。

本系统的意义如下:
\begin{itemize}
    \item 通过数据结构的优化和缩小优化方程的规模加速故事线生成算法
    \item 在故事线生成步骤中加入渲染与变形,使得交互框架更为灵活
    \item 开发一个半自动故事线生成系统,在GitHub上开源以供用户使用
\end{itemize}

