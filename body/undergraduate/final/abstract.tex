\cleardoublepage{}
\begin{center}
    \bfseries \zihao{3} 摘要
\end{center}
\quad\quad 故事线可视化是一种二维可视化形式,通过线条和时间轴呈现一组变化的实体关系,通常用来展示故事中角色及角色间的互动,在影视剧行业得到了广泛的利用。目前,该技术最主流的实现方式是手绘创作,其效果生动明了,但是过程复杂耗时。与之相对,全自动故事线可视化技术在性能方面取得了进步,但由于其利用解优化方程的全自动方式存在条件制约,呈现效果往往较为死板。本设计致力于兼顾手绘创作和全自动创作的优点,力求设计出能付诸实用的半自动创作系统。从理论半自动创作系统StoryFlow和iStoryline出发,我们通过改变数据结构提高了排序和布局压缩算法的效率,同时在生成过程中添加渲染和变形模块,从而提高系统的灵活性,让用户可以使用简单的操作便能探索复杂的设计空间。同时,本文也进行了用户调研,得到了十分积极的反馈,同时对未来的改进方向更加明晰。本文最后也给出了用户的实际使用案例,充分体现了本设计的可用性以及易用性。截至目前,相关系统在开源平台已受到相当关注,获得一定认可,说明该设计前景良好,具有很大的上升空间。


\textbf{关键字:故事线可视化; 优化问题; 半自动布局}


\cleardoublepage{}
\begin{center}
    \bfseries \zihao{3} Abstract
\end{center}
\quad\quad Storyline visualization is a kind of two-dimensional visualization. It presents a group of changing entity relations through lines and time axis, which is usually used to show the interaction between characters in the story, and is widely used in the film industry. At present, the most popular way to realize this technology is hand-painted. The output it provides is vivid and clear, but the process is complex and time-consuming. In contrast, the automatic storyline visualization technology has made progress in performance, but due to the constraints of its automatic way of solving optimization equations, the presentation effect is often more rigid. This design is committed to both the advantages of hand-painted creation and full-automatic creation, and strive to design a practical semi-automatic creation system. Starting from the theoretical semi-automatic creation system, StoryFlow and iStoryline, we improve the efficiency of sorting and layout compression algorithm by changing the data structure, and add rendering and deformation modules in the generation process, so as to improve the flexibility of the system, so that users can use simple operations to explore complex design space. This paper also carried out user research, obtaining positive feedback and orienting the further direction of improvement. At the end of this paper, the user's actual use case is also given, which fully reflects the utility and convenience of this design. Up to now, the system has gained attention in the open source platform, and has been recognized to a certain extent, which shows its promising prospect.


\textbf{Index Terms: Storylines, optimization problem, semi-automatic layout}