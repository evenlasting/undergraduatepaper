\cleardoublepage{}
\begin{center}
    \bfseries \zihao{3} 摘要
\end{center}
\quad\quad 故事线可视化是一种表达角色关系和情节的可视化形式。但是创作故事线是很复杂的,目前最主流的创作方式是使用手绘创作,手绘创作太复杂繁琐。全自动的创作方式又过于死板。半自动的创作方式可以极大地提高效率,同时保留灵活性。可惜半自动的创作方式只有相关论文进行研究并没有开源软件,本设计从半自动创作系统StoryFlow和iStoryline出发,通过改变数据结构提高了排序和布局压缩算法的效率,同时通过在生成过程中添加渲染和变形模块增加了系统的灵活性,让用户可以使用简单的操作来探索复杂的设计空间。文章也进行了用户调研,获得了用户的好评以及以后的修改方向。文章最后描述了一个用户的使用案例,证明了自己的可用性以及易用性。此系统在GitHub已收获15个star,之后还将就对脚本进行操作等存在的问题进行持续开发。

\textbf{关键字:故事线可视化; 优化问题; 半自动布局}


\cleardoublepage{}
\begin{center}
    \bfseries \zihao{3} Abstract
\end{center}
\quad\quad Storyline visualization is a form of visualization to express the relationship between characters and plots. But the creation of storyline is very complex. At present, the most mainstream way is to use hand-drawn creation, but this is too cumbersome. The automatic way of creation is too rigid. Semi-automatic creation can greatly improve efficiency while retaining flexibility. Unfortunately, the semi-automatic creation method is only studied in related papers. there is no open source software. This design starts from the semi-automatic creation system: StoryFlow and iStoryline. This design improves the efficiency of sorting and layout compression algorithm by changing the data structure, and increases the flexibility of the system by adding rendering and transforming modules in the generation process. So that users can use simple operations to explore the complex design space. This article also carried on the user study, obtained the user's high praise as well as the later revision direction. At the end of the paper, a user's use case is described to prove this system's usability and ease of use. This system has harvested 15 stars in GitHub, and will continue to develop.

\textbf{Index Terms: Storylines, optimization problem, semi-automatic layout}