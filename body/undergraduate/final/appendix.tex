\cleardoublepage

{
    \sectionnonum{附录}
    \appendixsubsecmajornumbering
    \subsection{故事线样图}
    以下为刘书含学长使用此系统制作的图片。值得注意的是,刘书含学长使用的版本中,变形交互由强化学习模块支持,其也为本人完成,但是超出了开题的范围,此文章不进行介绍。
    \begin{figure}[ht]
        \centering
        \includegraphics[width=1\linewidth]{body/storyline}
        \caption{\label{fig:append}用户使用此系统制作的故事线图,旨在说明实体之间的关系与交互}
    \end{figure}
    \clearpage
    \appendixsubsecmajornumbering

    \subsection{代码仓库}
    此系统的开源代码位于https://github.com/tangtan/iStoryline.js,为一个git仓库。所有者为唐谈学长,此仓库原先是为了存储他在2019年开发的iStoryline系统,其后我与唐谈学长、刘书含学长、吴欣科学长共同迭代开发此系统。我负责后端算法包的开发。本人的开发用户名为evenlasting或者lirenzhong,主要的开发分支位于dev以及layout。
    
    此系统的项目介绍以及开发文档均位于readme.md中,可供各位读者查阅。以及本文并未将测试文档写入,而是按照论文的格式撰写,一是因为仓库中已经含有此信息,二是因为此系统偏向科研,并且还没有到需要测试性能的阶段,仍需要继续迭代开发,才能最终作为产品发布。
    \clearpage
    \appendixsubsecmajornumbering

    \subsection{部署系统}
    此系统使用之江实验室GitLab的CI/CD功能自动部署在了K8集群上,网址为:plotthread.projects.zjvis.org。但是由于微软的官方.net docker镜像库经常失效,此系统可能会因为此原因自动部署失败而同步失效。

}